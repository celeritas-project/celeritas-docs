%%%%%%%%%%%%%%%%%%%%%%%%%%%%%%%%%%%%%%%%%%%%%%%%%%%%%%%%%%%%%%%%%%%%%%%%%%%%%%%
% File  : conference-papers/chep-2021/main.tex
%
% CHEP Conference paper
%%%%%%%%%%%%%%%%%%%%%%%%%%%%%%%%%%%%%%%%%%%%%%%%%%%%%%%%%%%%%%%%%%%%%%%%%%%%%%%
\documentclass{webofc}
\usepackage[varg]{txfonts}   % Web of Conferences font
\usepackage{hyperref}
\usepackage{color}
\usepackage{framed}
\usepackage{microtype}
\usepackage{xcolor}
\usepackage{xspace}
\usepackage{booktabs}

\usepackage[
    shortcuts,
    acronym,
    nonumberlist,
    nogroupskip,
    nopostdot]{glossaries}

\usepackage[
    detect-none,
    binary-units]{siunitx}

% >>> from pandoc
\usepackage{fancyvrb}
\newcommand{\VerbBar}{|}
\newcommand{\VERB}{\Verb[commandchars=\\\{\}]}
\DefineVerbatimEnvironment{Highlighting}{Verbatim}{commandchars=\\\{\},fontsize=\footnotesize}
\definecolor{shadecolor}{RGB}{248,248,248}
\newenvironment{Shaded}{\vspace{-.5\baselineskip}\begin{snugshade}}{\end{snugshade}\vspace{-.5\baselineskip}}
\newcommand{\AlertTok}[1]{\textcolor[rgb]{1.00,0.00,0.00}{\textbf{#1}}}
\newcommand{\AnnotationTok}[1]{\textcolor[rgb]{0.38,0.63,0.69}{\textbf{\textit{#1}}}}
\newcommand{\AttributeTok}[1]{\textcolor[rgb]{0.49,0.56,0.16}{#1}}
\newcommand{\BaseNTok}[1]{\textcolor[rgb]{0.25,0.63,0.44}{#1}}
\newcommand{\BuiltInTok}[1]{#1}
\newcommand{\CharTok}[1]{\textcolor[rgb]{0.25,0.44,0.63}{#1}}
\newcommand{\CommentTok}[1]{\textcolor[rgb]{0.38,0.63,0.69}{\textit{#1}}}
\newcommand{\CommentVarTok}[1]{\textcolor[rgb]{0.38,0.63,0.69}{\textbf{\textit{#1}}}}
\newcommand{\ConstantTok}[1]{\textcolor[rgb]{0.53,0.00,0.00}{#1}}
\newcommand{\ControlFlowTok}[1]{\textcolor[rgb]{0.00,0.44,0.13}{\textbf{#1}}}
\newcommand{\DataTypeTok}[1]{\textcolor[rgb]{0.56,0.13,0.00}{#1}}
\newcommand{\DecValTok}[1]{\textcolor[rgb]{0.25,0.63,0.44}{#1}}
\newcommand{\DocumentationTok}[1]{\textcolor[rgb]{0.73,0.13,0.13}{\textit{#1}}}
\newcommand{\ErrorTok}[1]{\textcolor[rgb]{1.00,0.00,0.00}{\textbf{#1}}}
\newcommand{\ExtensionTok}[1]{#1}
\newcommand{\FloatTok}[1]{\textcolor[rgb]{0.25,0.63,0.44}{#1}}
\newcommand{\FunctionTok}[1]{\textcolor[rgb]{0.02,0.16,0.49}{#1}}
\newcommand{\ImportTok}[1]{#1}
\newcommand{\InformationTok}[1]{\textcolor[rgb]{0.38,0.63,0.69}{\textbf{\textit{#1}}}}
\newcommand{\KeywordTok}[1]{\textcolor[rgb]{0.00,0.44,0.13}{\textbf{#1}}}
\newcommand{\NormalTok}[1]{#1}
\newcommand{\OperatorTok}[1]{\textcolor[rgb]{0.40,0.40,0.40}{#1}}
\newcommand{\OtherTok}[1]{\textcolor[rgb]{0.00,0.44,0.13}{#1}}
\newcommand{\PreprocessorTok}[1]{\textcolor[rgb]{0.74,0.48,0.00}{#1}}
\newcommand{\RegionMarkerTok}[1]{#1}
\newcommand{\SpecialCharTok}[1]{\textcolor[rgb]{0.25,0.44,0.63}{#1}}
\newcommand{\SpecialStringTok}[1]{\textcolor[rgb]{0.73,0.40,0.53}{#1}}
\newcommand{\StringTok}[1]{\textcolor[rgb]{0.25,0.44,0.63}{#1}}
\newcommand{\VariableTok}[1]{\textcolor[rgb]{0.10,0.09,0.49}{#1}}
\newcommand{\VerbatimStringTok}[1]{\textcolor[rgb]{0.25,0.44,0.63}{#1}}
\newcommand{\WarningTok}[1]{\textcolor[rgb]{0.38,0.63,0.69}{\textbf{\textit{#1}}}}
% <<< from pandoc

%%%%%%%%%%%%%%%%%%%%%%%%%%%%%%%%%%%%%%%%%%%%%%%%%%%%%%%%%%%%%%%%%%%%%%%%%%%%%%%
\setacronymstyle{long-short}
\makeglossaries

\newacronym{mc}{MC}{Monte Carlo}
\newacronym{lhc}{LHC}{Large Hadron Collider}
\newacronym{hpc}{HPC}{high performance computing}
\newacronym{hep}{HEP}{high energy physics}
\newacronym{olcf}{OLCF}{Oak Ridge Leadership Computing Facility}
\newacronym{ecp}{ECP}{Exascale Computing Project}
\newacronym{ornl}{ORNL}{Oak Ridge National Laboratory}
\newacronym{em}{EM}{electromagnetic}
\newacronym{fom}{FOM}{Figure of Merit}
\newacronym{sm}{SM}{streaming multiprocessor}

%%---------------------------------------------------------------------------%%
% Hyperref setup
\definecolor{CiteColor}{rgb}{0, 0, 0.55}
\definecolor{LinkColor}{rgb}{0.2, 0.2, 0.2}
\definecolor{URLColor}{rgb}{0.62745098, 0.1254902 , 0.94117647}
\hypersetup{
  linkcolor=LinkColor,
  citecolor=CiteColor,
  urlcolor=URLColor,
  colorlinks=true
}

% SI units
\sisetup{range-phrase = \text{--},
  group-separator={,},
  per-mode=symbol,
  group-minimum-digits=3,
  range-units=single}

% Commands
\newcommand{\Cpp}{C\texttt{++}\xspace}

%%---------------------------------------------------------------------------%%
\begin{document}

%%---------------------------------------------------------------------------%%
\title{Novel features and GPU performance analysis for EM particle transport in
  the Celeritas code%
  %
  \footnote{%
    This manuscript has been authored by
    UT-Battelle, LLC, under contract DE-AC05-00OR22725 with the US Department of
    Energy. The United States Government retains and the publisher, by accepting
    the article for publication, acknowledges that the United States Government
    retains a nonexclusive, paid-up, irrevocable, worldwide license to publish
    or reproduce the published form of this manuscript, or allow others to do
    so, for United States Government purposes. DOE will provide access to these
    results of federally sponsored research in accordance with the DOE Public
    Access Plan (http://energy.gov/downloads/doe-public-access-plan).%
  }%
}%
%%
\author{%
  \firstname{Seth R.} \lastname{Johnson}\inst{1}%
  \fnsep\thanks{\email{johnsonsr@ornl.gov}}
  %%
  \and
  \firstname{Stefano} \lastname{C. Tognini}\inst{1}
  %%
  %%
  \and
  \firstname{Philippe} \lastname{Canal}\inst{2}
  %%
  \and
  \firstname{Thomas} \lastname{Evans}\inst{1}
  %%
  \and
  \firstname{Soon Yung} \lastname{Jun}\inst{2}
  %%
  \and
  \firstname{Guilherme} \lastname{Lima}\inst{2}
  %%
  \and
  \firstname{Amanda} \lastname{Lund}\inst{3}
  %%
  \and
  \firstname{Vincent R.} \lastname{Pascuzzi}\inst{4}
}%
%%
\institute{%
  Oak Ridge National Laboratory
  \and
  Fermi National Accelerator Laboratory
  \and
  Argonne National Laboratory
  \and
  Lawrence Berkeley National Laboratory
}%
%%
\abstract{%
  Celeritas is a new computational transport code designed for high-performance
  simulation of high-energy physics detectors. This work describes some of its
  current capabilities and the design choices that enable the rapid development
  of efficient on-device physics. The abstractions that underpin the code design
  facilitate low-level performance tweaks that require no changes to the
  higher-level physics code. We evaluate a set of independent changes that
  together yield an almost 40\% speedup over the original GPU code for a net
  performance increase of $220\times$ for a single GPU over a single CPU
  running 8.4M tracks on a small demonstration physics app.
% CPU @ 8.4M tracks (d7c4168): 185.614546131
% GPU @ 8.4M tracks (d7c4168): 1.165428383
% GPU @ 8.4M tracks (489992c): 0.841786799
}%
%%
\maketitle

%%---------------------------------------------------------------------------%%
\section{Introduction}
\label{sec:introduction}

The new High Luminosity Large Hadron Collider (HL-LHC) Era of the LHC, along
with the upgrades in the detectors of its main experiments (CMS, ATLAS, ALICE,
and LHCb), will result in a steep rise in computing resource usage, far beyond
the expected availability within current funding scenarios
\cite{the_hep_software_foundation_roadmap_2019}. \ac{mc} simulations are a
large component of that expected increase, whose demand can be alleviated
through the use of \ac{hpc} hardware designed to use GPUs for better
performance at low power consumption.

The new \emph{Celeritas} particle transport code aims to close the gap between
the impending advanced architectures and the vast computational
requirements of the upcoming HEP detector campaigns. The main focus of
Celeritas is to implement full-fidelity high energy physics simulation of
LHC detectors on the advanced architectures that will form the backbone of
\ac{hpc} over the next decade.

A short-term goal for Celeritas is a proof-of-concept app and library for
simulating \ac{em} physics for photons and charged leptons on
geometry models currently used by the \ac{hep} community, starting with the CMS
detector.
This choice provides us the opportunity to verify the
performance gain of simulating \ac{em} showers, which are the most
computationally intensive part of a CMS event, on GPUs. Celeritas relies on the
CUDA-compatible VecGeom \cite{VecGeom:web} library to load and navigate existing
Geant4 \cite{geant4}-compatible geometry definitions.

This paper describes in \S\ref{code-architecture} some key software architecture
developments in Celeritas for enabling rapid implementation of high
performance, GPU-enabled physics. Using a demonstration physics app that
incorporates many of the these novel developments, we explore in
\S\ref{sec:miniapp} how changes to the computational kernel and to lower-level
Celeritas components effect run time performance on a contemporary GPU.

%%---------------------------------------------------------------------------%%
\section{Code Architecture}\label{code-architecture}

In the short term, Celeritas is designed as a standalone application
that transport particles exclusively on device. To support robust and
rapid unit testing, its components are designed to run natively in C++
on traditional CPUs regardless of whether CUDA is available for
on-device execution. This is accomplished both with simple macros that hide
CUDA function tags when not compiling CUDA code, and a new data model for
constructing and using complex hierarchical data on CPU and copying it to GPU.

Like other GPU-enabled \ac{mc} transport codes such as
Shift \cite{pandya_implementation_2016,hamilton_continuous-energy_2019},
the low-level component code used by transport kernels is designed so
that each particle track corresponds to a single thread, since particle
tracks once created are independent of each other. There is therefore
essentially no cooperation between individual threads, facilitating the
dual host/device annotation of most of Celeritas. The allocation of
secondary particles and the initialization of new tracks from these
secondaries both require CUDA-specific programming, but those components
are encapsulated so that physics code can correctly allocate secondaries both in
production code (on device) and in unit tests (on the host).

To support parallelizing our initial development over several team
members, and to facilitate refactoring and performance testing of code,
Celeritas uses a highly modular programming approach based on
composition rather than inheritance. As much as possible, each major
code component is built of numerous smaller components and interfaces
with as few other components as possible.

\subsection{Physics Interactor classes}

As an example of the granularity of classes, consider the sampling of
secondaries from a model. In contrast to the virtual
\texttt{G4VModel::sample\_secondaries} member function in Geant4, each
model in Celeritas defines an independent function-like object for
sampling secondaries. Each \texttt{Interactor} class is analogous a C++
standard library ``distribution'': the distribution parameters (including
particle properties, incident particle energy, incident direction, and
interacting element properties)
are class construction arguments, and the function-like
\texttt{Interactor::operator()} takes as its sole input a random number
generator and returns an \texttt{Interaction} object, which encodes the
change in state to the particle and the secondaries produced.

Table~\ref{tab:interactor} summarizes the models that are implemented in
Celeritas on device, as well as key challenges that were first encountered while
implementing each model. Future work will elaborate on the models and the novel
aspects of implementing them on GPU in Celeritas.
The mini-app and performance analysis in \S\ref{sec:miniapp} will focus on the
first interactor implemented, Klein--Nishina.

\begin{table}[htb]
  \centering
  \begin{tabular}{ll}
    \toprule %--------------------------------------------
    Model & Challenges \\
    \midrule %--------------------------------------------
    Klein--Nishina & Allocating secondaries \\
    Bethe--Heitler & Accessing element properties \\
    \(\textrm{e}^+ \to (\gamma, \gamma)\) & --- \\
    Moller/Bhabha scattering & Multiple energy distributions \\
    Livermore photoelectric & Accessing shell data \\
                            & Calculating cross sections on the fly \\
    \dots with atomic relaxation & Cascading electron vacancies \\
                                 & Dynamic number of secondaries \\
    \bottomrule %-----------------------------------------
  \end{tabular}
  \caption{Fully implemented model interactors in Celeritas.}
  \label{tab:interactor}
\end{table}

\subsection{Data model}\label{data-model}

Software for heterogeneous architectures must manage independent
\emph{memory spaces}. \emph{Host} allocations use \texttt{malloc} or
standard C++ library memory management, and the allocated data is
accessible only on the CPU. \emph{Device} memory is allocated with
\texttt{cudaMalloc} and is generally available only on the GPU. The CUDA
Unified Virtual Memory feature allows CUDA-allocated memory to be
automatically paged between host and device with a concomitant loss in
performance. Another solution to memory space management is a
portability layer such as Kokkos \cite{kokkos}, which manages the
allocation of memory and transfer of data between host and device using
a class
\texttt{Kokkos::View\textless{}class,\ MemorySpace\textgreater{}} which
can act like a \texttt{std::shared\_pointer} (it is reference counted),
a \texttt{std::vector} (it allocates and manages memory), and a
\texttt{std::span} (it can also provide a non-owning view to stored
data). A similar class has been developed for Celeritas but with the
design goal of supporting complicated heterogeneous data structures needed for
tabulated physics data, in contrast to Kokkos' focus on dense homogeneous linear
algebraic data.

The
\texttt{celeritas::Collection\textless{}class,\ Ownership,\ MemSpace\textgreater{}}
class manages data allocation and transfer between CPU and GPU with the
primary design goal of constructing deeply hierarchical data on host at
setup time and seamlessly copying to device. The templated
\texttt{class} must be trivially copyable -\/- either a fundamental data
type or a struct of such types. An individual item in a collection can
be accessed with \texttt{ItemId\textless{}class\textgreater{}}, which is a
trivially copyable but type-safe index; and a range of items (returned
as a \texttt{Span\textless{}class\textgreater{}}) can be accessed with a
trivially copyable \texttt{ItemRange\textless{}class\textgreater{}},
which is a container-like slice object containing start and stop
indices. Since \texttt{ItemRange} is trivially copyable and
\texttt{Collection}s have the same data layout on host and device, a set
of Collections that reference data in each other provide an effective,
efficient, and type-safe means of managing complex hierarchical data on
host and device.

As an example, consider material definitions (omitting isotopics for
simplicity), which contain three levels of indirection: an array of
materials has an array of pointers to elements and their fractions in
that material. In a traditional C++ code this could be represented as
\texttt{vector\textless{}vector\textless{}pair\textless{}Element*,\ double\textgreater{}\textgreater{}\textgreater{}}
with a separately allocated
\texttt{vector\textless{}Element\textgreater{}}. However, even with helper
libraries such as Thrust \cite{thrust} there is no direct CUDA analog to this
structure, because device memory management is limited to host code.
We choose not to use in-kernel \texttt{cudaMalloc} calls for
performance and portability considerations. Instead, Celeritas
represents the nested hierarchy as a set of \texttt{Collection} objects
bound together as a \texttt{MaterialParamsData} struct.
%
\begin{Shaded}
\begin{Highlighting}[]
\KeywordTok{struct}\NormalTok{ Element}
\NormalTok{\{}
    \DataTypeTok{int}\NormalTok{            atomic\_number;}
\NormalTok{    units::AmuMass atomic\_mass;}
\NormalTok{\};}

\KeywordTok{struct}\NormalTok{ MatElementComponent}
\NormalTok{\{}
\NormalTok{    ItemId\textless{}Element\textgreater{} element;}
    \DataTypeTok{real\_type}\NormalTok{       fraction;}
\NormalTok{\};}

\KeywordTok{struct}\NormalTok{ Material}
\NormalTok{\{}
    \DataTypeTok{real\_type}\NormalTok{   number\_density;}
    \DataTypeTok{real\_type}\NormalTok{   temperature;}
\NormalTok{    MatterState matter\_state;}
\NormalTok{    ItemRange\textless{}MatElementComponent\textgreater{} components;}
\NormalTok{\};}

\KeywordTok{template}\NormalTok{\textless{}Ownership W, MemSpace M\textgreater{}}
\KeywordTok{struct}\NormalTok{ MaterialParamsData}
\NormalTok{\{}
    \KeywordTok{template}\NormalTok{\textless{}}\KeywordTok{class}\NormalTok{ T\textgreater{}} \KeywordTok{using}\NormalTok{ Items = celeritas::Collection\textless{}T, W, M\textgreater{};}

\NormalTok{    Items\textless{}Element\textgreater{}             elements;}
\NormalTok{    Items\textless{}MatElementComponent\textgreater{} elcomponents;}
\NormalTok{    Items\textless{}Material\textgreater{}            materials;}
    \DataTypeTok{unsigned} \DataTypeTok{int}\NormalTok{               max\_elcomponents;}

    \KeywordTok{template}\NormalTok{\textless{}Ownership W2, MemSpace M2\textgreater{}}
\NormalTok{    MaterialParamsData\& }\KeywordTok{operator}\NormalTok{=(}\AttributeTok{const}\NormalTok{ MaterialParamsData\textless{}W2, M2\textgreater{}\& other);}
\NormalTok{\};}
\end{Highlighting}
\end{Shaded}

A
\texttt{MaterialParamsData\textless{}Ownership::value,\ MemSpace::host\textgreater{}}
is constructed incrementally on the host (each
\texttt{Items\textless{}class\textgreater{}} in that template
instantiation is a thin wrapper to a \texttt{std::vector}), then copied
to a
\texttt{MaterialParamsData\textless{}Ownership::value,\ MemSpace::device\textgreater{}}
(where each \texttt{Items\textless{}class\textgreater{}} is a separately
managed \texttt{cudaMalloc} allocation) using the templated assignment
operator. The definition of that operator simply assigns each element
from the \texttt{other} instance. For primitive data such as
\texttt{max\_elcomponents}, the value is simply copied; for
\texttt{Items} the templated \texttt{Collection::operator=} performs a
host-to-device transfer under the hood.

To access the data on device, a
\texttt{MaterialParamsData\textless{}Ownership::const\_reference,\ MemSpace::device\textgreater{}}
(where each \texttt{Items\textless{}class\textgreater{}} is a
\texttt{Span\textless{}class\textgreater{}} pointing to device memory)
is constructed using the sample assignment operator from the
\texttt{\textless{}value,\ device\textgreater{}} instance of the data.
The \texttt{const\_reference} and \texttt{reference} instances of a
Collection are trivially copyable and can be passed as kernel arguments
or saved to global memory.

The first material in a
\texttt{MaterialParamsData\textless{}Ownership::const\_reference,\ MemSpace::device\textgreater{}\ data}
instance can be accessed as:
\begin{Shaded}
\begin{Highlighting}[]
\AttributeTok{const}\NormalTok{ Material\& m = data.materials[ItemId\textless{}Material\textgreater{}(}\DecValTok{0}\NormalTok{)];}
\end{Highlighting}
\end{Shaded}
%
A view of its elemental component data is:
%
\begin{Shaded}
\begin{Highlighting}[]
\NormalTok{Span\textless{}}\AttributeTok{const}\NormalTok{ MatElementComponent\textgreater{} els = data.elcomponents[material.components];}
\end{Highlighting}
\end{Shaded}
%
And the elemental properties of the first constituent of the material
are:
%
\begin{Shaded}
\begin{Highlighting}[]
\AttributeTok{const}\NormalTok{ Element\& el = data.elements[data.element];}
\end{Highlighting}
\end{Shaded}

In practice, the \texttt{MaterialParamsData} itself is an implementation
detail constructed by the host-only class \texttt{MaterialParams} and
used by the device-compatible class \texttt{MaterialView} and
\texttt{ElementView}, which encapsulate access to the material and
element data. In Celeritas, \texttt{View} objects are to
\texttt{Collection} as \texttt{std::string\_view} is to
\texttt{std::vector\textless{}char\textgreater{}}.

\subsection{States and parameters}\label{states-and-parameters}

The Celeritas data model is careful to separate persistent shared
``parameter'' data from dynamic local ``state'' data, as there will
generally one independent state per GPU thread. To illustrate the
difference between parameters and states, consider the calculation of
the Lorentz factor \(\gamma\) of a particle, which is a function of both
the rest mass \(mc^2\)---which is constant for all particles of the same
type but is not a fundamental constant nor the same for all
particles---and the kinetic energy \(K\). It is a parameterized
expression \(\gamma(m;K) = 1 + \frac{K}{mc^2}\). Celeritas
differentiates shared data such as \(m\) (parameters, shortened to
\texttt{Params}) from state data particular to a single track such as
kinetic energy \(K\) or particle type (\texttt{State}). Inside transport
kernels, the \texttt{ParticleTrackView} class combines the parameter and
state data with the local GPU thread ID to encapsulate the fundamental
properties of a track's particle properties -\/- its rest mass, charge,
and kinetic energy, as well as derivative properties such the magnitude
of its relativistic momentum.

In Celeritas, a particle track is not a single object nor a struct of
arrays. Instead, sets of classes (Params plus State) define aspects of a
track, each of which is accessed through a \texttt{TrackView} class.
Table~\ref{tab:modules} shows the current track attributes independently
implemented in Celeritas.
%
\begin{table}[htb]
  \centering
  \begin{tabular}{@{}lll@{}}
\toprule
Module & State & Params\tabularnewline
\midrule
Particle & Kinetic energy and particle ID & Properties for each particle
type \\
Material & Current material ID & Material densities, elements,
etc. \\
Geometry & Position, direction, volume ID, NavState & Geometry
description \\
Sim & Track ID, Event ID, time & --- \\
Physics & Distance to interaction, & Models, processes, \\
& cached cross sections & cross section tables \\
\bottomrule
  \end{tabular}
  \caption{Existing groups of per-track state and parameter data in Celeritas.}
  \label{tab:modules}
\end{table}

In addition to the per-track attributes, each hardware thread
(which may correspond to numerous tracks in different events over the lifetime
of the simulation) has a persistent random number state initialized at the start
of the program.

\subsection{On-device allocation}\label{on-device-allocation}

One requirement for transporting particles in electromagnetic showers is
the efficient allocation and construction of secondary particles. The
number of secondaries produced during an interaction varies according to
the physics process, random number generation, and other properties.
This implies a large variance in the number of secondaries produced from
potentially millions of tracks undergoing interactions in parallel on
the GPU.

To enable runtime dynamic allocation of secondary particles, we have
authored a function-like
\texttt{StackAllocator\textless{}class\textgreater{}} templated class
that uses a large on-device allocated array with a fixed capacity along
with an atomic addition to unambiguously reserve one or more items in
the array. The call argument to a \texttt{StackAllocator} takes as an
argument the number of elements to allocate, and if allocation is
successful, it uses placement new to default-initialize each element and
returns a pointer to the first element. If the capacity is exceeded
during the allocation (or by a parallel thread also in the process of
allocating), a null pointer is returned. Figure~\ref{fig:secondary} describes
the allocation algorithm, including the error conditions necessary to ensure
that the size of the allocated elements is correct even in the case of an
overflow.
%
\begin{figure}[htb]
  \centering
  \includegraphics[width=4in]{fig/secondary-allocation}
  \caption{Flowchart of the stack allocation algorithm.}
  \label{fig:secondary}
\end{figure}

To accommodate large numbers of secondaries on potentially limited GPU
memory, we define a \texttt{Secondary} class that carries the minimal
amount of information needed to reconstruct it from the parent track,
rather than as a full-fledged track:

\begin{Shaded}
\begin{Highlighting}[]
\KeywordTok{struct}\NormalTok{ Secondary}
\NormalTok{\{}
\NormalTok{    ParticleId       particle\_id\{\};}
\NormalTok{    units::MevEnergy energy\{zero\_quantity()\};}
\NormalTok{    Real3            direction;}
\NormalTok{\};}
\end{Highlighting}
\end{Shaded}

The final aspect of GPU-based secondary allocation is how to gracefully
handle an out-of-memory condition without crashing the simulation
\emph{or} invalidating its reproducibility. This can be accomplished by
ensuring that no random numbers are sampled before allocating storage
for the secondaries, and by adding a external loop
over the interaction
kernel (Fig.~\ref{fig:interaction}) to reallocate extra secondary space or
process secondaries so
that all interactions can successfully complete in the exceptional case
where the secondary storage space is exceeded.

\begin{figure}[htb]
  \centering
  \includegraphics[width=3in]{fig/interaction-secondaries}
  \caption{Flowchart for an interaction kernel wrapped in a host-side loop for
  processing secondaries.}
  \label{fig:interaction}
\end{figure}


%%---------------------------------------------------------------------------%%
\section{Mini-App Results}
\label{sec:miniapp}

Using the Celeritas components, we constructed a demonstration app to verify a
simple test problem against Geant4 \cite{geant4} results. It is the simplest physical
simulation we can run, with photon-only transport and a single
Compton scattering process using the Klein--Nishina model.  It has a single
infinite material (aluminum) and a 100 MeV monodirectional point source.

The stepping kernel is parallel over particle tracks, with one launch per step,
and ``dead'' tracks ignored. Interaction lengths are sampling with
uniform-in-log-energy cross section calculations with linear interpolation. The
particle states include position and directions that are updated with each step.
Secondaries are allocated and constructed as part of each interaction, but they
are immediately killed and their energy deposited locally. Each energy
deposition event, whether from an absorbed electron or a cutoff photon,
allocates (using a \texttt{StackAllocator<Hit>} instance) a detector hit and
writes the deposited energy, position, direction, time, and track ID to global
memory.

An additional kernel processes the allocated vector of detector hits into
uniformly spaced detector bins. A final kernel performs a reduction on the
``alive'' state of particles to determine whether the simulation should
terminate. These helper kernels are included in the timings reported below.

The same code components were used to build GPU and CPU versions of the same
stepping process, although the CPU version steps through one track at a time
rather than many tracks in parallel.  Figure~\ref{fig:baseline} gives the
baseline performance of the two versions. The host code uses a single core of an
Intel ``Cascade Lake'' Xeon processor running at 2.3 GHz, compiled with GCC 8.3
and \texttt{-O3 -march=skylake-avx512 -mtune=skylake-avx512}. The device code
uses a single Nvidia Tesla V100 running at 1.53 GHz, compiled with CUDA 10.1 and
\texttt{-O3 --use\_fast\_math}.

\begin{figure}[htb]
  \centering
  \includegraphics[width=3.5in]{fig/cpu-gpu-comparison}
  \caption{Performance comparison of the CPU and original GPU versions of the
  Celeritas code. The ``total'' GPU plot includes the extra kernel launches for
  processing detector hits and the number of living tracks.}
  \label{fig:baseline}
\end{figure}

A roofline analysis (Fig.~\ref{fig:roofline}) shows that the interaction kernel
is not limited by raw memory bandwidth or floating point performance. XXX @tme
please add a couple of sentences.

\begin{figure}[htb]
  \centering
  \includegraphics[width=3.5in]{fig/roofline}
  \caption{Roofline plot showing the interaction kernel performance
    plotted against the theoretical memory bandwidth and arithmetic performance
    limitations. The 1/2/4 points are the number of particle tracks processed by
    a single GPU thread.}
  \label{fig:roofline}
\end{figure}

Although this mini-app is far simpler than one that supports the full range of
physics needed for EM shower simulation, it may be instructive to see how
standard recommendations for performance enhancement affect the total runtime.
We experimented with the demo app and underlying Celeritas components by
independently making each of the following changes:
\begin{itemize}
  \item \emph{Removing the ``grid striding'' wherein a single GPU thread can
    transport multiple tracks sequentially.} Preliminary studies showed
    decreased performance by transporting multiple tracks per thread, so in all
    the results shown here only one track is used per thread. Removing the grid
    striding should reduce the register pressure for the kernel by eliminating
    an unused runtime variable.
  \item \emph{Copying track states into local variables.} Operating on the
    position, direction, and time locally (rather than as pointers to global
    memory) should remove potential aliasing issues and improve potential
    compiler optimizations.
  \item \emph{Copying the random number generator state into a thread-local
    variable.}
    Like with the other track states, the RNG state (XORWOW) is accessed
    directly through global memory. The CURAND documentation notes
    that to increase performance, the generator state can be operated on locally
    but stored in global memory between kernel launches. If the increased
    register usage spills into local memory, then at least the memory usage will
    be coalesced.
  \item \emph{Using a struct-of-arrays rather than array-of-structs for particle
    data.} For the sake of expediency, the \texttt{ParticleTrackState} data is a
    struct with a particle type ID (\texttt{unsigned int}) and an energy
    (\texttt{double}). Conventional CUDA kernels obtain higher memory
    bandwidth when data accesses are ``coalesced,'' which will be more likely
    when each component is a contiguous array. We should note that the
    \texttt{ParticleTrackView} abstraction completely hides this implementation
    change from the tracking kernel and the physics code.
  \item \emph{Preallocating one secondary per interaction.} Rather than using
    the dynamic \texttt{SecondaryAllocator} and its atomic add, preallocate a
    single \texttt{Secondary} as part of the \texttt{Interaction} result.
    Physics kernels that allocate more than one secondary per interaction will
    still need the dynamic allocation, but simpler kernels will no longer need
    the atomic, at the cost of slightly increased memory pressure.
  \item \emph{Splitting the single step kernel into two kernels, one for movement
    and one for interaction.} Smaller kernels tend to have lower register usage
    and therefore higher occupancy.
  \item \emph{Using a 32-bit instead of 64-bit integer for the stack allocator.}
    Smaller data reduces memory bandwidth, and CUDA operations tend to be
    inherently faster for 32-bit types such as the native unsigned int and
    single-precision floats.
\end{itemize}
Figure~\ref{fig:speedup} shows the performance of each separate change in the
list above, as well as a combination of all changes.
%
\begin{figure}[htb]%
  \centering%
  \includegraphics[width=3.5in]{fig/speedups}%
  \caption{Incremental (thin colored lines) and cumulative (thick gray line)
  speed up for changes to the mini-app and data structures.}%
  \label{fig:speedup}%
\end{figure}

Aggregating the speedup values for cases with more than $10^6$ tracks (mean and
$1\sigma$ given for $\mbox{speedup}=\mbox{original}/\mbox{adjusted} - 1$, in
percent):
\begin{itemize}
  \item \emph{Removing the ``grid striding''} had a slightly negative impact
    ($-1.3\% \pm 0.3\%$). Diagnostics showed
    that removing grid striding did in fact increase occupancy from 37.5\% to
    50\%, but this did not translate to improved performance.
  \item \emph{Copying track states into local variables} had a small positive
    effect ($+3.1\% \pm 0.6\%$). Analysis of the PTX assembly code showed that
    with the cost of a few extra arithmetic operations, the improved kernel
    decreased the number of global memory loads from 41 to 29, because storing
    the values locally informs the compiler that there are no aliasing effects
    that might cause hidden dependencies between the data. However, this a 25\%
    reduction in global memory accesses resulted only in a 3\% speedup.
  \item \emph{Copying the RNG states into a local variable} had essentially no effect
    ($-0.8\% \pm 0.5\%$). Examining the emitted PTX code shows almost no change
    between the original and modified version. This suggests that the extensive
    use of inline functions in Celeritas allows the compiler to determine that
    the RNG state (a struct of a type not used anywhere else in the code) cannot
    be aliased or modified by external functions calls, and thus does not have
    to be reloaded between subsequent calls to the RNG functions.
  \item \emph{Using a struct-of-arrays rather than array-of-structs for particle
    data} had zero significant effect. Coalescing memory access in this demo app
    appears unimportant: the minor reduction in memory transactions is swamped
    by the cost of the inherently random accesses for the cross section
    calculation and dynamic allocations.
  \item \emph{Preallocating one secondary per interaction} improved performance
    more than any other change thus far ($+13.4\% \pm 0.3\%$). There was one
    fewer atomic operation (the detector ``hits'' still remained) and a decrease
    in global
    memory accesses from loading the allocated secondary to process.
  \item \emph{Splitting the single step kernel into two kernels} had the most
    negative effect ($-7.7\% \pm 0.9\%$). This is not unexpected because each
    kernel launch has additional overhead, and each independent kernel has to
    reload data from global memory that might otherwise be stored in registered.
    Still, a 10\% drop in performance might provide a substantial gain in code
    flexibility and extensibility.
  \item \emph{Using a 32-bit instead of 64-bit allocation size} had the most
    positive individual change ($+28.0\% \pm 0.1\%$). In conjunction with the
    secondary preallocation result, this suggests that the atomic operations may
    be the single most expensive aspect of this simple demonstration kernel.
\end{itemize}
%
The overall speedup of $+38\%$ suggests that the faster and fewer atomic
operations negate the performance drop of the atomic-heavy interaction split
kernel.

Of these results, the lack of performance gain for a substantial increase in
occupancy is perhaps the most surprising.
Loosely stated, occupancy measures the ratio of threads that can be
\emph{active} to the maximum number of threads on a \ac{sm},
which is the core hardware computational component of a CUDA card. Higher
occupancy can hide latency to improve overall kernel performance.
To explore the performance implications of higher occupancy, we recompiled the
fully ``optimized'' kernel with the \verb|--maxrregcount N| NVCC
compiler option to constrain kernel register usage,
which is the limiting factor for the CUDA \ac{sm} occupancy for the demo
interactor kernels. Figure~\ref{fig:occupancy} demonstrates that a higher
kernel occupancy does not always translate to improved performance.
%
\begin{figure}[htb]
  \centering%
  \includegraphics[width=3.5in]{fig/occupancy}%
  \caption{Performance ramifications of forcing the register size to be smaller
  for higher occupancy. The blue line is total solve time, the dark red line is
the register usage, and the light red line is local memory usage including
memory spills.}%
  \label{fig:occupancy}
\end{figure}

%%---------------------------------------------------------------------------%%
\section{Conclusion}

The Celeritas project is nascent and a work in progress, with its
early developmental phase establishing a foundation of core classes and
design patterns to facilitate development of an efficient, GPU-enabled particle
transport code for high energy physics. The initial results from a simple but
nontrivial GPU physics simulation showed an ``unoptimized'' factor of about
$160\times$ performance improvement over a single-CPU version of the same
simulation with identical underpinning components. A set of potential
optimizations was independently examined, and their cumulative effect boosted
the speedup to about $220\times$ relative to the CPU version. Since adding
additional kernels increased the overall runtime, we expect a complete EM
physics app to have a smaller speedup than this mini-app.  The ease of
implementing and testing each change promises that the infrastructure in place
will enable similar optimizations in the future full-featured physics
application.

The next step in the Celeritas development will incorporate VecGeom geometry
transport, multiple materials, and multiple physics models, including
continuous-slowing-down processes. Future work will present the design choices
made to enable these additional features, as well as more CPU comparisons and
GPU performance analysis.

%%---------------------------------------------------------------------------%%
\section{Acknowledgements}

Work for this paper was supported by Oak Ridge National Laboratory (ORNL), which is managed and operated by UT-Battelle, LLC, for the U.S. Department of Energy (DOE) under Contract No. DEAC05-00OR22725 and by Fermi National Accelerator Laboratory, managed and operated by Fermi Research Alliance, LLC under Contract No. DE-AC02-07CH11359 with the U.S. Department of Energy.
%%
This research was supported by the Exascale Computing
Project (ECP), project number 17-SC-20-SC. The ECP is a collaborative effort of
two DOE organizations, the Office of Science and the National Nuclear Security
Administration, that are responsible for the planning and preparation of a
capable exascale ecosystem---including software, applications, hardware,
advanced system engineering, and early testbed platforms---to support the
nation's exascale computing imperative.

%%---------------------------------------------------------------------------%%
\bibliography{references}
%%---------------------------------------------------------------------------%%
\end{document}
%%%%%%%%%%%%%%%%%%%%%%%%%%%%%%%%%%%%%%%%%%%%%%%%%%%%%%%%%%%%%%%%%%%%%%%%%%%%%%%
% end of conference-papers/chep-2021/main.tex
%%%%%%%%%%%%%%%%%%%%%%%%%%%%%%%%%%%%%%%%%%%%%%%%%%%%%%%%%%%%%%%%%%%%%%%%%%%%%%%

%%%%%%%%%%%%%%%%%%%%%%%%%%%%%%%%%%%%%%%%%%%%%%%%%%%%%%%%%%%%%%%%%%%%%%%%%%%%%%%
% File  : conference-papers/chep-2021/main.tex
%
% CHEP Conference paper
%%%%%%%%%%%%%%%%%%%%%%%%%%%%%%%%%%%%%%%%%%%%%%%%%%%%%%%%%%%%%%%%%%%%%%%%%%%%%%%
\documentclass{webofc}
\usepackage[varg]{txfonts}   % Web of Conferences font
\usepackage{hyperref}
\usepackage{color}

\usepackage[
    shortcuts,
    acronym,
    nonumberlist,
    nogroupskip,
    nopostdot]{glossaries}

\usepackage[
    detect-none,
    binary-units]{siunitx}
%%%%%%%%%%%%%%%%%%%%%%%%%%%%%%%%%%%%%%%%%%%%%%%%%%%%%%%%%%%%%%%%%%%%%%%%%%%%%%%
\setacronymstyle{long-short}
\makeglossaries

\newacronym{mc}{MC}{Monte Carlo}
\newacronym{lhc}{LHC}{Large Hadron Collider}
\newacronym{hpc}{HPC}{high performance computing}
\newacronym{hep}{HEP}{high energy physics}
\newacronym{olcf}{OLCF}{Oak Ridge Leadership Computing Facility}
\newacronym{ecp}{ECP}{Exascale Computing Project}
\newacronym{ornl}{ORNL}{Oak Ridge National Laboratory}
\newacronym{em}{EM}{electromagnetic}
\newacronym{fom}{FOM}{Figure of Merit}
%%---------------------------------------------------------------------------%%
% Hyperref setup
\definecolor{CiteColor}{rgb}{0, 0, 0.55}
\definecolor{LinkColor}{rgb}{0.2, 0.2, 0.2}
\definecolor{URLColor}{rgb}{0.62745098, 0.1254902 , 0.94117647}
\hypersetup{
  linkcolor=LinkColor,
  citecolor=CiteColor,
  urlcolor=URLColor,
  colorlinks=true
}

% SI units
\sisetup{range-phrase = \text{--},
  group-separator={,},
  per-mode=symbol,
  group-minimum-digits=3,
  range-units=single}
%%---------------------------------------------------------------------------%%
\begin{document}

%%---------------------------------------------------------------------------%%
\title{Celeritas: Toward GPU-based particle transport for detector simulations
    in LHC experiments}
\author{
    \firstname{Seth} \lastname{Johnson}\inst{1}\fnsep
    \thanks{\email{johnsonsr@ornl.gov}}
    %%
    \and
    \firstname{Philippe} \lastname{Canal}\inst{2}
    %%
    \and
    \firstname{Thomas} \lastname{Evans}\inst{1}
    %%
    \and
    \firstname{Soon Yung} \lastname{Jun}\inst{2}
    %%
    \and
    \firstname{Guilherme} \lastname{Lima}\inst{2}
    %%
    \and
    \firstname{Amanda} \lastname{Lund}\inst{3}
    %%
    \and
    \firstname{Vincent} \lastname{Pascuzzi}\inst{4}
    %%
    \and
    \firstname{Stefano} \lastname{Tognini}\inst{1}
}

\institute{Oak Ridge National Laboratory
    \and
    Fermi National Accelerator Laboratory
    \and
    Argonne National Laboratory
    \and
    Lawrence Berkeley National Laboratory
}

\abstract{
    Celeritas is awesome.
}

\maketitle

%%---------------------------------------------------------------------------%%
\section{Introduction}
\label{sec:introduction}

Upgrades to the \ac{lhc} and its detectors (including CMS, ALICE, and ATLAS)
demand a commensurate increase in modeling and simulation capacity that is out
of reach of traditional software but can be alleviated through the use of
advanced \ac{hpc} hardware that use GPUs for improved performance at low power
consumption.  Our objective is to provide the needed modeling capacity using a
new application \emph{Celeritas} that performs fast and accurate \ac{mc}
particle transport simulations on GPUs. Celeritas is designed to complement, not
replace, Geant4 and ultimately satisfy the detector response requirements as
defined in Ref.~\cite{the_hep_software_foundation_roadmap_2019} using  the
advanced architectures that will form the backbone of \ac{hpc} over the next
decade.

%% Need the basic project plan here

In this paper we will focus primarily on the components of the Celeritas
software architecture (Sec.~\ref{sec:code-arch}) that will enable high
performance on GPU hardware.  Particular attention is paid to the construction
of the Celeritas data model (Sec.~\ref{sec:data-model}), as memory management on
heterogeneous devices is critical to code performance, particularly for random
access algorithms such as \ac{mc}.  Improvements to the VecGeom geometry package to enable \ac{mc} transport on GPUs are discussed in Sec.~\ref{sec:vecgeom}.  Finally, we show some preliminary performance results on a mini-application (mini-app) in Sec.~\ref{sec:miniapp}.

%%---------------------------------------------------------------------------%%
\section{Code Architecture}
\label{sec:code-arch}

%%-------------------%%
\subsection{Data Model}
\label{sec:data-model}

%%-----------------------%%
\subsection{Physics Models}
\label{sec:physics-models}

%%---------------------------------------------------------------------------%%
\section{VecGeom Geometry Package}
\label{sec:vecgeom}

%%---------------------------------------------------------------------------%%
\section{Mini-App Results}
\label{sec:miniapp}

%%---------------------------------------------------------------------------%%
\bibliography{references}
%%---------------------------------------------------------------------------%%
\end{document}
%%%%%%%%%%%%%%%%%%%%%%%%%%%%%%%%%%%%%%%%%%%%%%%%%%%%%%%%%%%%%%%%%%%%%%%%%%%%%%%
% end of conference-papers/chep-2021/main.tex
%%%%%%%%%%%%%%%%%%%%%%%%%%%%%%%%%%%%%%%%%%%%%%%%%%%%%%%%%%%%%%%%%%%%%%%%%%%%%%%

%%------------------------------------%%
\subsubsection{Klein--Nishina}

% Seth

%%------------------------------------%%
\subsubsection{Bethe--Heitler}

% Vince
The Bethe--Heitler model~\cite{Bethe:1934za} describes the stopping power of
matter for highly-energetic particles.
As such, it is used in Geant4 for a range of electromagnetic
physics processes, \textit{e.g.} photon conversions to electron-positon or muon-
antimuon pairs and nuclear scattering.
Several variants of Bethe--Heitler are implemented for cross-section
calculations of such processes, each having different energy thresholds on the
incident particle to describe different energy regimes, and may also take into
account additional effects.
For instance, \texttt{G4BetheHeitlerModel} is applicable below
$\sim100$~MeV while \texttt{G4BetheHeitler5DModel} is used for incident photon
energies $>100$~MeV;
the latter is the only photon conversion variant of the model that takes into
account recoil momentum.
Celeritas currently has implemented a Bethe--Heitler model for energies
$< 100$~MeV to describe photon conversions to electron-positron pairs.
These codes follow closely \texttt{G4BetheHeitlerModel} to calculate
the impact parameter of the incident photon and apply Coulomb corrections for
sampling of the final-state kinematics.
Since recoil effects are not considered, energy is conserved exactly but
momentum is not.
%%------------------------------------%%
\subsubsection{Positron annihilation}

$\textrm{e}^+ \to (\gamma, \gamma)$
% Soon

%%------------------------------------%%
\subsubsection{Moller/Bhabha scattering}

% Stefano
In Geant4, both Moller and Bhabha scatterings are sampled by a single
class method, which defines the criteria based on the incident particle,
performs the interaction (Moller for electrons and Bhabha for positrons), and
calculates the final incident and secondary particle states. Celeritas
follows a similar structure, where a MollerBhabha interactor class is
responsible for performing the interaction and calculating the final particle
states. However, in Celeritas, the sampling algorithms for each scattering
process are broken down into two separate classes, each of which responsible
solely for executing the sampling loop and returning the fraction of the energy
transferred during the scattering process to the main MollerBhabha interactor,
which then proceeds to calculate the final particle states. This higher
compartmentalization improves code testing and maintainability, and provides
an opportunity to test code performance in the future by splitting the current
mechanism into two different models and templating the interactor by particle
type.

%%------------------------------------%%
\subsubsection{Livermore photoelectric}

The photoelectric effect is simulated in Celeritas using the Livermore
photoelectric model, which combines parameterized and tabulated partial cross
sections, tabulated total cross sections, and atomic shell data to sample the
ionized shell and determine the energy of the emitted photoelectron. The
subsequent deexcitation of the atom from its excited state to the ground state
through a cascade of radiative and non-radiative transitions is implemented as
a separate class that can be used by any process that produces a vacancy in an
atomic shell. The atomic relaxation implementation uses transition energies and
probabilities from the Evaluated Atomic Data Library to simulate both
fluorescence and Auger electron emission. The data required to simulate the
photoelectric effect and atomic deexcitation is loaded directly from the
G4EMLOW data files into memory by separate specialized reader classes.

While macroscopic cross sections for most processes can be interpolated from
tabulated data exported from Geant4, Geant4 calculates these cross sections on
the fly for the photoelectric effect below 200 keV and for positron
annihilation at all energies. Therefore, on-the-fly macroscopic cross section
calculation has also been implemented in Celeritas for both of these processes.

